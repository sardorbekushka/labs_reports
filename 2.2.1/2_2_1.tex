\documentclass[12pt,a4paper]{article}
\usepackage{amsmath}
\usepackage{mathtext}
\usepackage{icomma}
\usepackage{amsfonts}
\usepackage{amssymb}
\usepackage[utf8]{inputenc}
\usepackage[T1,T2A]{fontenc}
\usepackage[english, russian]{babel}
\usepackage{graphicx}
\usepackage[left=2cm,right=2cm,top=2cm,bottom=2cm]{geometry}
\usepackage{calc}
\usepackage{wrapfig}
\usepackage{setspace}
\usepackage{indentfirst}
\usepackage{subfigure}
\usepackage[table,xcdraw]{xcolor}


\title{
Отчет о выполнении лабораторной работы 2.2.1 \\
Исследование взаимной диффузии газов.}

\author{Исламов Сардор, группа Б02-111}
\date{19 апреля 2022 г.}
\begin{document}	
\maketitle

\subparagraph*{Аннотация.}
В ходе выполнения данной работы определена зависмость разности концентраций гелия и воздухе в сосуде от времени. 
По результатам измерений определен коэффицент взаимной диффузии газов.
\subsection*{Теоретическое введение}
Рассмотрим процесс выравнивания концентрации. Пусть концентрации одного из компонентов смеси в сосудах $V_1$ и $V_2$ равны $n_1$ и
$n_2$. Плотность диффузионного потока любого компонента (т. е. количество вещества, проходящее в единицу времени через единичную поверхность) определяется законом Фика:
$$j=-D\frac{\partial n}{\partial x},$$ где $D$ — коэффициент взаимной диффузии газов, а $j$ - плотность потока частиц.

В нашем случае ввиду того что, а) объем соединительной трубки мал по сравнению с объемами сосудов, б) концентрацию газов внутри каждого сосуда можно считать постоянной по всему объему. Диффузионный поток в любом сечении трубки одинаков. Поэтому, $$J=-DS\frac{n_1-n_2}{l}.$$

Обозначим через $\Delta n_1$ и $\Delta n_2$ изменения концентрации в объемах
$V_1$ и $V_2$ за время $\Delta t$. Тогда $V_1 \Delta n_1$ равно изменению количества компонента в объеме $V_1$, а $V_2 \Delta n_2$ — изменению количества этого компонента в $V_2$. Из закона сохранения вещества следует, что $V_1n_1+V_2n_2 = const$, откуда $V_1 \Delta n_1 = -V_2\Delta n_2.$ Эти изменения происходят вследствие диффузии, поэтому: $$V_1\Delta n_1=-V_2\Delta n_2.$$

С другой стороны $V_1\Delta n_1=J\Delta t$ и $V_1\frac{dn_1}{dt}=-DS\frac{n_1-n_2}{l}.$ Аналогично $V_2\frac{dn_2}{dt}=DS\frac{n_1-n_2}{l}$

Тогда $$\frac{d(n_1-n_2)}{dt}=-\frac{n_1-n_2}{l} \frac{V_1+V_2}{V_1V_2}.$$

Проинтегрируем и получим, что $$n_1-n_2=(n_1-n_2)_0 e^{-t/\tau},$$ где $(n_1-
n_2)_0$ — разность концентраций в начальный момент времени, 
\begin{equation}
	\tau=\frac{V_1V_2}{V_1+V_2}\frac{l}{SD}
\end{equation}
Для измерения концентраций в данной установке применяются датчики теплопроводности $Д_1$, $Д_2$ (см. рис. 1) используется зависимость теплопроводности газовой смеси от ее состава.
Для измерения разности концентраций газов используется мостовая схема (рис. 1). Здесь $Д_1$ и $Д_2$ — датчики теплопроводности, расположенные в сосудах $V_1$ и $V_2$. Сопротивления $R_1, R_2$ и $R$ служат для установки прибора на нуль (балансировка моста). В одну из диагоналей моста включен гальванометр, к другой подключается небольшое постоянное напряжение. Мост балансируется при заполнении сосудов (и датчиков) одной и той же смесью.

При заполнении сосудов смесями различного состава возникает «разбаланc» моста. При незначительном различии в составах смесей показания гальванометра, подсоединённого к диагонали моста, будут пропорциональны разности концентраций примеси. В процессе диффузии
разность концентраций убывает по экспоненте, и значит по тому же закону изменяются во времени показания гальванометра 
\begin{equation}
	U=U_0 \exp(-t/\tau)
\end{equation}
\subsection*{Эксперементальная установка}
\begin{figure}[htp]
	\centering
	\includegraphics[scale=0.5]{scheme1.png}
	\includegraphics[scale=0.6]{scheme2.png}
	\caption{Схема установки}
\end{figure}
Схема установки изображена на рис. 1. Там же показана схема электрических соединений и конструкция многоходового крана $K_6$

Установка состоит из двух сосудов $V_1$ и $V_2$ соединенных краном $К_3$, форвакуумного насоса Ф.Н. с выключателем $Т$, манометра $M$ и системы напуска гелия, включающей в себя краны $К_6$ и $К_7$. Кран $К_5$ позволяет соединять форвакуумный насос либо с установкой, либо с атмосферой. Между форвакуумным насосом и краном $К_5$ вставлен предохранительный баллон П.Б., защищающий кран $К_5$ и установку при неправильной эксплуатации ее от попадания форвакуумного масла из насоса Ф.Н. Сосуды $V_1$ и $V_2$ и порознь и вместе можно соединять как с системой напуска гелия, так и с форвакуумным насосом. Для этого служат краны $К_1$, $К_2$, $К_4$ и $К_5$. Манометр  $M$
регистрирует давление газа, до которого заполняют тот или другой
сосуды.

Для сохранения гелия, а также для уменьшения неконтролированного попадания гелия в установку (по протечкам в кране $К_6$) между
трубопроводом подачи гелия и краном $К_6$ поставлен металлический
кран $К_7$. Его открывают только на время непосредственного заполнения установки гелием. Все остальное время он закрыт.

В силу того, что в сосуд требуется подавать малое давление гелия,
между кранами $К_7$ и $К_4$ стоит кран $К_6$, снабженный дозатором. Дозатор - это маленький объем, который заполняют до давления гелия в трубопроводе, а затем уже эту порцию гелия с помощью крана $К_6$ впускают в установку.

Описание схемы электрического соединения. $Д_1$ и $Д_2$ — сопротивления проволок датчиков парциального давления, которые составляют одно плечо моста. Второе плечо моста составляют сопротивления $r_1$, $R_1$ и $r_2$, $R_2$. $r_1 \ll R_1$, $r_2 \ll R_2$, $R_1$ и $R_2$ спаренные, их подвижные контакты находятся на общей оси. Оба они исполь- зуются для грубой регулировки моста. Точная балансировка моста выполняется потенциометром R. Последовательно с гальванометром $Г$, стоящим в диагонали моста, поставлен магазин сопротивлений $MR$. Когда мост балансируют, магазин сопротивлений выводят на ноль. В процессе же составления рабочей смеси в сосудах $V_1$ и $V_2$ мост разбалансирован. Чтобы не сжечь при этом гальванометр, магазин $MR$ ставят на максимальное сопротивление.

\subsection*{Ход работы}
\begin{enumerate}
\item Включим питание электрической схемы установки рубильником $B$. Откроем краны $К_1$, $K_2$, $К_3$. Перепишем параметры установки: $$V_1 = V_2 = V = 800 \pm 5 \; см^{3}, \; \frac{L}{S} = 15.0 \pm 0.1 \;см^{-1}$$
\item Очистим установку от всех газов, откачав установку до давления $\approx 0.1\ торр $.
\item Напустим в установку воздух до рабочего давления и сбалансируем мост. 
\item Снова очистим установку и заполним её рабочей смесью: в сосуде $V_2$ должен быть воздух ($P \approx 1.75P_\Sigma$), а в сосуде $V_1$ — гелий ($P \approx 0.2 P_\Sigma$).
\item Проведём измерения и повторим процедуру для нескольких рабочих давлений в диапазоне 40--300 торр. Данные записаны в таблице 1.
\begin{table}[htp]
	\centering
	\begin{tabular}{|c|c|c||c|c|c||c|c|c|}
		\hline
		\multicolumn{9}{|c|}{P, торр}\\
		\hline
		\multicolumn{3}{|c||}{40.00} & \multicolumn{3}{|c||}{103.60} & \multicolumn{3}{|c|}{253.60}\\
		\hline
		t, c & U, мВ&$\ln{U\over U_0}$&t, c & U, мВ&$\ln{U\over U_0}$&t, c & U, мВ&$\ln{U\over U_0}$\\
		\hline
		0  &  13.8  &  0.0  &  0  &  14.12  &  0.0  &  0  &  14.5  &  0.0 \\ 
		\hline
	   10  &  13.5  &  0.02  &  20  &  13.88  &  0.02  &  60  &  14.1  &  0.03 \\ 
		\hline
	   20  &  13.21  &  0.04  &  40  &  13.62  &  0.04  &  120  &  13.73  &  0.05 \\ 
		\hline
	   30  &  12.93  &  0.07  &  60  &  13.37  &  0.05  &  180  &  13.39  &  0.08 \\ 
		\hline
	   40  &  12.62  &  0.09  &  80  &  13.12  &  0.07  &  240  &  13.08  &  0.1 \\ 
		\hline
	   50  &  12.35  &  0.11  &  100  &  12.88  &  0.09  &  300  &  12.78  &  0.13 \\ 
		\hline
	   60  &  12.09  &  0.13  &  120  &  12.65  &  0.11  &  360  &  12.51  &  0.15 \\ 
		\hline
	   70  &  11.83  &  0.15  &  140  &  12.42  &  0.13  &  420  &  12.25  &  0.17 \\ 
		\hline
	   80  &  11.56  &  0.18  &  160  &  12.2  &  0.15  &  480  &  11.98  &  0.19 \\ 
		\hline
	   90  &  11.31  &  0.2  &  180  &  11.98  &  0.16  &  540  &  11.73  &  0.21 \\ 
		\hline
	   100  &  11.07  &  0.22  &  200  &  11.77  &  0.18  &  600  &  11.49  &  0.23 \\ 
		\hline
	   110  &  10.82  &  0.24  &  220  &  11.6  &  0.2  &  660  &  11.26  &  0.25 \\ 
		\hline
	   120  &  10.59  &  0.26  &  240  &  11.36  &  0.22  &  720  &  11.03  &  0.27 \\ 
		\hline
	   130  &  10.37  &  0.29  &  260  &  11.15  &  0.24  &  780  &  10.82  &  0.29 \\ 
		\hline
	   140  &  10.16  &  0.31  &  280  &  10.98  &  0.25  &  840  &  10.6  &  0.31 \\ 
		\hline
	   150  &  9.92  &  0.33  &  300  &  10.77  &  0.27  &  900  &  10.39  &  0.33 \\ 
		\hline
	   160  &  9.7  &  0.35  &  320  &  10.58  &  0.29  &  960  &  10.19  &  0.35 \\ 
		\hline
	   170  &  9.5  &  0.37  &  340  &  10.4  &  0.31  &  1020  &  9.99  &  0.37 \\ 
		\hline
	   180  &  9.3  &  0.39  &  360  &  10.23  &  0.32  &  1080  &  9.82  &  0.39 \\ 
		\hline
	   190  &  9.11  &  0.42  &  380  &  10.04  &  0.34  &  1140  &  9.63  &  0.41 \\ 
		\hline
	   200  &  8.92  &  0.44  &  400  &  9.87  &  0.36  &  1200  &  9.46  &  0.43 \\ 
		\hline
	   210  &  8.73  &  0.46  &  420  &  9.71  &  0.37  &  1260  &  9.27  &  0.45 \\ 
		\hline
	   220  &  8.53  &  0.48  &  440  &  9.54  &  0.39  &  1320  &  9.1  &  0.47 \\ 
		\hline
	   230  &  8.35  &  0.5  &  460  &  9.38  &  0.41  &  1380  &  8.93  &  0.48 \\ 
		\hline
	   240  &  8.17  &  0.52  &  480  &  9.23  &  0.43  &  1440  &  8.78  &  0.5 \\ 
		\hline
	   250  &  8.01  &  0.54  &  500  &  9.07  &  0.44  &  1500  &  8.62  &  0.52 \\ 
		\hline
	   260  &  7.84  &  0.57  &  520  &  8.9  &  0.46  &  1560  &  8.45  &  0.54 \\ 
		\hline
	   270  &  7.65  &  0.59  &  540  &  8.78  &  0.48  &  1620  &  8.3  &  0.56 \\ 
		\hline
	\end{tabular}
	\caption{Показания, снятые с установки}
	\begin{flushright}
		\includegraphics[scale=0.2]{sign.jpeg}
	\end{flushright}
\end{table}

\item Построим графики по полученным значениям (рис. 2) и найдем коэффиценты наклона по МНК:
	\begin{figure}[htp]
		\centering
		\includegraphics[scale=0.7]{lnut_dis.png}
		\caption{Графики зависимости $\ln{U\over U_0}$ от времени}
	\end{figure}

	\begin{equation*}
		k_1 = (2187 \pm 2) \cdot 10^{-6} c^{-1},\ k_2 = (893 \pm 2)\ \cdot 10^{-6} c^{-1},\ k_3 = (358 \pm 3) \cdot 10^{-6} c^{-1}.
	\end{equation*}

	\item Теперь расчитаем коэффициенты взаимной диффузии и построим график зависимости $D({1\over P})$ (рис. 3).

	Из (1) и (2) следует $D = k {VL \over 2S}$:
	\begin{equation*}
		D_1 = 13.12 \pm 0.12\ {см^2 \over c},\ D_2 = 5.36 \pm 0.05\ {см^2 \over c},\ D_3 = 2.15 \pm 0.03\ {см^2 \over c}
	\end{equation*}

	\begin{figure}[htp]
		\centering
		\includegraphics[scale=0.7]{D1P.png}
		\caption{График зависимости $D({1\over P})$}
	\end{figure}

	Отложив расчитанные точки на графике по МНК получаем коэффицент наклона $k = 529 \pm 6\ {см^2 \cdot торр\over с} = 7.05\pm 0.08\ {кг\cdot м \over с^3} \Rightarrow  D = 0.71 \pm 0.01\ {см^2\over c}\ (\varepsilon = 2\%)$.
	При этом табличное значение составляет $D = 0.62\ {см^2 \over c}$.

	\item Теперь расчитаем длину свободного пробега атомов гелия в воздухе: 
	\begin{equation*}
		\lambda_{He} = 3D\sqrt{\pi \mu \over 8RT} = (1.71 \pm 0.02) \cdot 10^{-7}\ м\ (\varepsilon = 2\%)\ для\ T = 300K.
	\end{equation*}
	И среднее сечение столкновения частиц гелия с воздухом:
	\begin{equation*}
		\sigma_{He-возд} = {1\over \lambda_{He} n_\Sigma} = {k_Б T \over \lambda_{He} P} = (4.84 \pm 0.04) \cdot 10^{-19} м^2\ (\varepsilon = 2\%).
	\end{equation*}
\end{enumerate}

\subsection*{Вывод}
В ходе выполнения данной работы определена зависимость разности концентрации частиц гелия и воздуха в смеси от времени, по полученным результатам расчитаны коэффициенты взаимной диффузии воздуха и гелия при разных давлениях.
После определения зависимости между полученными коэффицентами и давлением получен коэффицент взаимной диффузии гелия и воздуха при атмосферном давлении для используемой установки $D = 0.71 \pm 0.01\ {см^2\over c}\ (\varepsilon = 2\%)$. 
Табличное значение составляет $D = 0.62\ {см^2\over c}$.
Также при комнатной температуре $T = 300K$ расчитаны длина свободного пробега атомов гелия в воздухе $\lambda_{He} = (1.71 \pm 0.02) \cdot 10^{-7}\ м\ (\varepsilon = 2\%)$ и среднее сечение столкновения частиц гелия с воздухом $\sigma_{He-возд} = (4.84 \pm 0.04) \cdot 10^{-19} м^2\ (\varepsilon = 2\%).$
\end{document}