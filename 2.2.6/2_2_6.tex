\documentclass[12pt,a4paper]{article}
\usepackage{amsmath}
\usepackage{mathtext}
\usepackage{icomma}
\usepackage{amsfonts}
\usepackage{amssymb}
\usepackage[utf8]{inputenc}
\usepackage[T1,T2A]{fontenc}
\usepackage[english, russian]{babel}
\usepackage{graphicx}
\usepackage[left=2cm,right=2cm,top=2cm,bottom=2cm]{geometry}
\usepackage{calc}
\usepackage{wrapfig}
\usepackage{setspace}
\usepackage{indentfirst}
\usepackage{subfigure}
\usepackage[table,xcdraw]{xcolor}


\title{
Отчет о выполнении лабораторной работы 2.2.6 \\
Определение энергии активации по температурной завсимости вязкости жидкости.}

\author{Исламов Сардор, группа Б02-111}
\date{5 апреля 2022 г.}

\begin{document}

\maketitle

\subparagraph*{Аннотация.} В ходе работы измерены скорости свободного падения шариков в жидкости при разной температуре.
По полученным значениям по закону Стокса расчитана вязкость и энергия активации исследуемой жидкости (глицерин).

\subsection*{Теоретическое введение}

В отличие от твёрдых тел, жидкости обладают «рыхлой» структурой. 
В них имеются свободные места — «дырки», благодаря чему молекулы могут перемещаться, покидая своё место и занимая одну из соседних дырок. 
Таким образом, молекулы медленно перемещаются внутри жидкости, пребывая часть времени около определённых мест равновесия и образуя картину меняющейся со временем пространственной решётки. 
На современном языке принято говорить, что в жидкости присутствует ближний, но не дальний порядок, расположение молекул упорядочено в небольших объёмах, но порядок перестаёт замечаться при увеличении расстояния.


Для того, чтобы перейти в новое состояние, молекула должна преодолеть участки с большой потенциальной энергией, превышающей среднюю тепловую энергию молекул. 
Для этого тепловая энергия молекул должна (вследствие флуктуации) увеличиться на некоторую величину $W$, называемую энергией активации. 
Вследствие этого переходы молекул из одного положения равновесия в другое происходят сравнительно редко и тем реже, чем больше энергия активации.

Отмеченный характер движения молекул объясняет как медленность диффузии в жидкостях, так и большую (по сравнению с газами) их вязкость. 
В газах вязкость объясняется происходящим при тепловом движении молекул переносом количества направленного движения. 
В жидкостях такие переходы существенно замедлены. 
Количество молекул, имеющих энергии больше $W$, в соответствии с формулой Больцмана экспоненциально зависит от $W$. 
Температурная зависимость вязкости жидкости выражается формулой:
\begin{equation}
    \eta \sim Ae^{W/kT}
\end{equation}

Из формулы (1) следует, что вязкость жидкости при повышении температуры должна резко уменьшаться. 
Если отложить на графике логарифм вязкости $\ln \eta$ в зависимости от $1/T$, то согласно (1) должна получиться прямая линия, по угловому коэффициенту которой можно определить энергию активации молекулы $W$ исследуемой жидкости. 
Экспериментальные исследования показывают, что в небольших температурных интервалах эта формула неплохо описывает изменение вязкости с температурой. 
При увеличении температурного интервала согласие получается плохим, что представляется вполне естественным, поскольку формула (1) выведена при очень грубых предположениях.

Для исследования температурной зависимости вязкости жидкости в данной работе используется метод Стокса, основанный на измерении скорости свободного падения шарика в жидкости. 
Суть его заключается в следующем.

На всякое тело, двигающееся в вязкой жидкости, действует сила сопротивления. 
В общем случае величина этой силы зависит от многих факторов: от вязкости жидкости, от формы тела, от характера обтекания и т. д. 
Стоксом было получено строгое решение задачи о ламинарном обтекании шарика безграничной жидкостью. 
В этом случае сила сопротивления $F$ определяется формулой
\begin{equation}
    F = 6\pi \eta rv
\end{equation}
где $\eta$ — вязкость жидкости, $v$ — скорость шарика, $r$ — его радиус.

Рассмотрим свободное падение шарика в вязкой жидкости. 
На шарик действуют три силы: сила тяжести, архимедова сила и сила вязкости, зависящая от скорости.

Найдём уравнение движения шарика в жидкости. 
По второму закону Ньютона:

\begin{equation}
    Vg(\rho - \rho_ж) - 6\pi \eta r v = V \rho {dv \over dt}
\end{equation}

где $V$ — объём шарика, $\rho$ — его плотность, $\rho_ж$ — плотность жидкости, $g$ — ускорение свободного падения. 
Решая это уравнение, найдём
\begin{equation}
    v(t) = v_{уст} - [v_{уст} - v(0)]e^{t/\tau}
\end{equation}
В формуле (4) приняты обозначения: $v(0)$ — скорость шарика в момент начала его движения в жидкости,
\begin{equation}
    v_{уст} = {Vg(\rho - \rho_ж)\over 6 \pi \eta r} = \frac 2 9 gr^2 {\rho - \rho_ж \over \eta},\ \tau = {V\rho \over 6\pi \eta r} = \frac 2 9 {r^2 \rho \over \eta}
\end{equation}

\subsection*{Экспериментальная установка}
\begin{wrapfigure}{r}{0.5\linewidth}
    \centering
    \includegraphics[scale = 0.2]{scheme.png}
    \caption{Схема установки}    
\end{wrapfigure}
Для измерений используется стеклянный цилиндрический сосуд $В$, наполненный исследуемой жидкостью (глицерин). 
Диаметр сосуда $\approx 3$ см, длина $\approx 40$ см. 
На стенках сосуда нанесены две метки на расстоянии $10$ см друг от друга. 
Верхняя метка должна располагаться ниже уровня жидкости с таким расчётом, чтобы скорость шарика к моменту прохождения этой метки успевала установиться. 
Измеряя расстояние между метками с помощью линейки, а время падения с помощью секундомера, определяют скорость шарика $v_{уст}$. 
Сам сосуд $B$ помещён в рубашку $D$, омываемую водой из термостата. 
При работающем термостате температура воды в рубашке $D$, а потому и температура жидкости 12 равна температуре воды в термостате.

Радиусы шариков измеряются микроскопом. 
Для каждого шарика рекомендуется измерить несколько различных диаметров и вычислить среднее значение. 
Такое усреднение целесообразно, поскольку в работе используются шарики, форма которых может несколько отличаться от сферической. 

Опыты проводятся при нескольких температурах в интервале от комнатной до $50–60 ^oC$.

\subsection*{Результаты измерений и обработка данных}
Расстояние l между рисками = $10\pm 0.1$ см.
$\rho_{стекло} = 2.5\ г/см^3,\ \rho_{сталь} = 7.8\ г/см^3$

Несколько первых замеров времени произведем с нескольких секундомеров для вычисления погрешностей.
$\sigma_t = 0.15$ с.

В таблице при каждой температуре результаты представлены сначала для двух стеклянных шариков, далее для двух стальных.

$t_1\ и\ t_2$ - время прохождения шариком соответственно расстояний между первыми двумя и следующими двумя рисками (в первой таблице также $t_1'\ и \ t_2'$ - время, замеренное вторым секундомером)

Стеклянные шарики будем считать симметричными. 
Для одного стального проведём несколько измерений диаметра с разных сторон для вычисления случайной погрешности.

\begin{table}[htp]
    \centering
    \begin{tabular}[]{|c|c|c|c|c|c|c|}
        \hline
        $r_1,\ мм$ & $r_2,\ мм$ & $r_3,\ мм$ &$\langle r \rangle,\ мм$ & $\sigma_r^{сл},\ мм$ & $\sigma_r^{сист},\ мм $ & $\sigma_r,\ мм$ \\
        \hline
        0.425 & 0.425 & 0.45 & 0.433& 0.008&0.0025&0.008\\
        \hline
    \end{tabular}    
\end{table}


Число Рейнольдса будем расчитывать по формуле $Re = {vr\rho_ж / \eta}$.
В случае, если оно оказалось больше 0.5 коэффициент вязкости пересчитывается по формуле:
\begin{equation}
    \eta = \frac 2 9 g r^2 {\rho - \rho_{ж} \over (1 + 2.4\ r/R)v_{уст}}
\end{equation}

Занесем все значения в табл. 1

\begin{table}[htp]
    \centering

    \begin{tabular}[]{|c|c|c|c|c|c|c|c|c|c|c|}
        \hline
        \multicolumn{11}{|c|}{$T = 25.00^oC$, $\rho = 1.26\ г/см^3$}\\
        \hline
        $r$, мм & $t_1$, c & $t_2$, c & $t_1'$, c & $t_2'$, c & $\langle t\rangle $, с & $\eta$,& $\sigma_{\eta}$,& $Re$ & $\tau$, мс & $S,$\\
        &&&&&& мПа$\cdot$с & мПа$\cdot$с &&&$10^{-3}\ мм$\\
        \hline
        1.05  &  24.4 & 24.68 & 24.76 & 24.7 &  24.64  &  734.19  &  8.6  &  0.073  &  0.42  &  1.71 \\ 
        \hline
        1.075  &  24.56 & 24.91 & 24.94 & 24.48 &  24.72  &  772.29  &  9.04  &  0.071  &  0.42  &  1.69 \\ 
        \hline
        0.45  &  25.43 & 25.54 & 25.19 & 25.65 &  25.45  &  734.82  &  8.92  &  0.03  &  0.08  &  0.3 \\ 
        \hline
        0.35  &  40.86 & 41.05 & 41.68 & 40.6 &  41.05  &  716.91  &  8.31  &  0.015  &  0.05  &  0.12 \\ 
        \hline

    \end{tabular}

    \begin{tabular}[]{|c|c|c|c|c|c|c|c|c|}
        \hline
        \multicolumn{9}{|c|}{$T = 30.00^oC$, $\rho = 1.26\ г/см^3$}\\
        \hline
        $r$, мм & $t_1$, c & $t_2$, c & $\langle t\rangle $, с & $\eta$,& $\sigma_{\eta}$,& $Re$ & $\tau$, мс & $S$\\
        &&&& мПа$\cdot$с & мПа$\cdot$с &&&$10^{-3}\ мм$\\
        \hline
        1.05  &  17.49 & 17.36 &  17.42  &  519.31  &  6.86  &  0.146  &  0.59  &  3.41 \\ 
        \hline
        1.0  &  17.05 & 17.28 &  17.16  &  464.0  &  6.17  &  0.158  &  0.6  &  3.52 \\ 
        \hline
        0.45  &  19.49 & 19.8 &  19.64  &  567.17  &  7.42  &  0.051  &  0.1  &  0.51 \\ 
        \hline
        0.433  &  21.48 & 21.43 &  21.46  &  573.51  &  7.31  &  0.044  &  0.09  &  0.43 \\ 
        \hline
    
        \hline
        \multicolumn{9}{|c|}{$T = 35.00^oC$, $\rho = 1.25\ г/см^3$}\\
        \hline
        $r$, мм & $t_1$, c & $t_2$, c & $\langle t\rangle $, с & $\eta$,& $\sigma_{\eta}$,& $Re$ & $\tau$, мс & $S$\\
        &&&& мПа$\cdot$с & мПа$\cdot$с &&&$10^{-3}\ мм$\\
        \hline
        1.025  &  10.98 & 10.64 &  10.81  &  309.49  &  5.3  &  0.383  &  0.94  &  8.72 \\ 
        \hline
        1.025  &  11.46 & 11.49 &  11.48  &  328.52  &  5.41  &  0.34  &  0.89  &  7.74 \\ 
        \hline
        0.45  &  11.97 & 12.01 &  11.99  &  346.69  &  5.69  &  0.135  &  0.16  &  1.35 \\ 
        \hline
        0.45  &  12.55 & 12.67 &  12.61  &  364.62  &  5.81  &  0.122  &  0.15  &  1.22 \\ 
        \hline
        
        \hline
        \multicolumn{9}{|c|}{$T = 40.00^oC$, $\rho = 1.25\ г/см^3$}\\
        \hline
        $r$, мм & $t_1$, c & $t_2$, c & $\langle t\rangle $, с & $\eta$,& $\sigma_{\eta}$,& $Re$ & $\tau$, мс & $S$\\
        &&&& мПа$\cdot$с & мПа$\cdot$с &&&$10^{-3}\ мм$\\
        \hline
        1.025  &  8.06 & 8.11 &  8.09  &  213.93  &  4.88  &  0.685  &  1.36  &  16.87 \\ 
        \hline
        1.05  &  8.08 & 8.18 &  8.13  &  225.32  &  5.13  &  0.661  &  1.36  &  16.72 \\ 
        \hline
        0.35  &  12.34 & 12.6 &  12.47  &  218.12  &  3.55  &  0.161  &  0.16  &  1.25 \\ 
        \hline
        0.4  &  10.39 & 10.25 &  10.32  &  235.77  &  4.27  &  0.205  &  0.19  &  1.83 \\ 
        \hline
    
        \hline
        \multicolumn{9}{|c|}{$T = 45.00^oC$, $\rho = 1.25\ г/см^3$}\\
        \hline
        $r$, мм & $t_1$, c & $t_2$, c & $\langle t\rangle $, с & $\eta$,& $\sigma_{\eta}$,& $Re$ & $\tau$, мс & $S$\\
        &&&& мПа$\cdot$с & мПа$\cdot$с &&&$10^{-3}\ мм$\\
        \hline
        1.025  &  6.07 & 5.81 &  5.94  &  157.17  &  4.62  &  1.268  &  1.86  &  31.26 \\ 
        \hline
        1.025  &  5.81 & 5.83 &  5.82  &  154.0  &  4.61  &  1.321  &  1.9  &  32.56 \\ 
        \hline
        0.425  &  6.65 & 6.74 &  6.7  &  172.67  &  4.29  &  0.46  &  0.29  &  4.34 \\ 
        \hline
        0.425  &  7.42 & 7.5 &  7.46  &  192.4  &  4.38  &  0.37  &  0.26  &  3.5 \\ 
        \hline
    
        \hline
        \multicolumn{8}{|c|}{$T = 50.00^oC$, $\rho = 1.24\ г/см^3$}\\
        \hline
        $r$, мм & $t_1$, c & $t_2$, c & $\langle t\rangle $, с & $\eta$,& $\sigma_{\eta}$,& $Re$ & $\tau$, мс & $S$\\
        &&&& мПа$\cdot$с & мПа$\cdot$с &&&$10^{-3}\ мм$\\
        \hline
        1.025  &  4.05 & 4.36 &  4.2  &  112.15  &  4.5  &  2.491  &  2.58  &  61.39 \\ 
        \hline
        1.0  &  4.49 & 4.27 &  4.38  &  111.4  &  4.29  &  2.353  &  2.47  &  56.48 \\ 
        \hline
        0.425  &  5.49 & 5.81 &  5.65  &  141.15  &  4.18  &  0.639  &  0.35  &  6.24 \\ 
        \hline
        0.35  &  6.67 & 6.52 &  6.6  &  112.39  &  2.92  &  0.57  &  0.3  &  4.55 \\ 
        \hline
    \end{tabular}
    \caption{Результаты измерений}
    \begin{flushright}
        \includegraphics[scale=0.1]{sign.jpeg}
    \end{flushright}
\end{table}

Путь релаксации в каждом опыте много меньше расстояния от уровня жидкости до первой риски, поэтому скорость шариков можно считать установившейся.

Построим теперь графиик зависимости $\ln \eta\ (1/T)$ (рис. 2)

\begin{figure}[htp]
    \centering
    \includegraphics[scale=0.73]{ln_n_1_T2.png}
    \caption{Зависимость $\ln \eta (1/T)$}
\end{figure}

Угловой коэффициент $\alpha$ для стеклянных и стальных шариков найдём по МНК: 

$\alpha_{стекло} = 6977.5\pm 142.5\ K,\ \alpha_{сталь} = 6852.4\pm283.7\ K \Rightarrow \alpha = 6914\pm 316\ K$

В таком случае энергия активации составит $W = k\alpha = (9.54 \pm 0.44) \cdot 10^{-20}\ Дж\ (\varepsilon \approx 5\%)$. 

\subsection*{Вывод}
В ходе данной работы в большинстве опытов подтверждена срправедливость формулы Стокса. 
Также определены время и путь релаксации, подтверждающие установление скорости шариков к началу измерений
Установлена зависимость вязкости глицерина от температуры. По значениям вязкости при малых температурах и сравнении их с табличными, можно предположить, что в работе использовался 99\% водный раствор глицерина:\\
при температуре $T = 25^oC$ $\eta = 772\ мПа\cdot с$ (для 100\% раствора $\eta = 942\ мПа\cdot с$), полученное значение $\eta_1 = (739\pm 18)\ мПа \cdot с\ (\varepsilon = 2\%)$; \\
при $T = 30^oC$ $\eta = 510\ мПа\cdot с$ (для 100\%$\eta = 662\ мПа\cdot с$), полученное значение $\eta_2 = (531\pm 14)\ мПа \cdot с\ (\varepsilon = 3\%)$

Исходя из полученных значений установлена зависмость вязкости от температуры, и по ней найдена энергия активации глицерина $W = (9.54\pm 0.44)\cdot 10^{-20}$ Дж $(\varepsilon = 5\%)$.
\end{document}