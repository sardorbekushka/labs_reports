\documentclass[12pt,a4paper]{article}
\usepackage{amsmath}
\usepackage{mathtext}
\usepackage{icomma}
\usepackage{amsfonts}
\usepackage{amssymb}
\usepackage[utf8]{inputenc}
\usepackage[T1,T2A]{fontenc}
\usepackage[english, russian]{babel}
\usepackage{graphicx}
\usepackage[left=2cm,right=2cm,top=2cm,bottom=2cm]{geometry}
\usepackage{calc}
\usepackage{wrapfig}
\usepackage{setspace}
\usepackage{indentfirst}
\usepackage{subfigure}
\usepackage[table,xcdraw]{xcolor}


\title{
Отчет о выполнении лабораторной работы 2.4.1 \\
Определение теплоты испарения жидкости.}

\author{Исламов Сардор, группа Б02-111}
\date{29 марта 2022 г.}

\begin{document}

\maketitle

\subparagraph*{Аннотация.} В данной работе установлена зависимость давления насыщенного пара жидкости от температуры;
по полученным данным с помощью уравнения Клапейрона–Клаузиуса вычислены теплоты испарения.

\subsection*{Теоретическое введение}
Испарением называется переход вещества из жидкого в газообразное состояние. 
Оно происходит на свободной поверхности жидкости. 
При испарении с поверхности вылетают молекулы, образуя над ней пар. 
Для выхода из жидкости молекулы должны преодолеть силы молекулярного сцепления. 
Кроме того, при испарении совершается работа против внешнего давления $P$, поскольку объём жидкости меньше объёма пара. 
Не все молекулы жидкости способны совершить эту работу, а только те из них, которые обладают достаточной кинетической энергией. 
Поэтому переход части молекул в пар приводит к обеднению жидкости быстрыми молекулами, т. е. к её охлаждению. 
Чтобы испарение проходило без изменения температуры, к жидкости нужно подводить тепло. 
Количество теплоты, необходимое для изотермического испарения одного моля жидкости при внешнем давлении, равном упругости её насыщенных паров, называется молярной теплотой испарения (парообразования).

Теплоту парообразования жидкостей можно измерить непосредственно при помощи калориметра. 
Такой метод, однако, не позволяет получить точных результатов из-за неконтролируемых потерь тепла, которые трудно сделать малыми. 
В настоящей работе для определения теплоты испарения применен косвенный метод, основанный на формуле Клапейрона–Клаузиуса:

\begin{equation}
    \frac{dP}{dT} = {L \over T(V_2-V_1)}
\end{equation}

Здесь $P$ — давление насыщенного пара жидкости при температуре $T$, $T$ — абсолютная температура жидкости и пара, $L$ — теплота испарения жидкости, $V_2$ — объём пара, $V_1$ — объём жидкости. 
Найдя из опыта $\frac{dP}{dT},\ T,\ V_2\ и\ V_1$, можно определить $L$ путём расчёта. 
Величины $L,\ V_2\ и\ V_1$ в формуле (1) должны относиться к одному и тому же количеству вещества; мы будем относить их к одному молю.

В нашем приборе измерения производятся при давлениях ниже атмосферного. 
В этом случае задача существенно упрощается.

Из табличных значений видно, что $V_1$ не превосходит 0,5\% от $V_2$. 
При нашей точности опытов величиной  $V_1$ в (1) можно пренебречь.
Обратимся теперь к $V_2$, которое в дальнейшем будем обозначать просто $V$. Объём $V$ связан с давлением и температурой уравнением Ван-дер-Ваальса:
\begin{equation}
    \left(p + \frac{a}{V^2}\right)\left(V - b\right) = RT
\end{equation}

Из таблицы можно видеть, что $b$ одного порядка с $V_1$. 
В уравнении Ван-дер-Ваальса величиной $b$ следует пренебречь. 
Пренебрежение членом $\frac a {V_2}$ по сравнению с $P$ вносит ошибку менее 3\%. 
При давлении ниже атмосферного ошибки становятся ещё меньше. 
Таким образом, при давлениях ниже атмосферного уравнение Ван-дер-Ваальса для насыщенного пара мало отличается от уравнения Клапейрона. 
Положим поэтому
\begin{equation}
    V = {RT \over P}
\end{equation}

Подставляя (3) в (1), пренебрегая $V_1$ и разрешая уравнение относительно $L$, найдём
\begin{equation}
    L = {RT^2 \over P} {dP \over dT} = -R {d(\ln P) \over d(1/T)}.
\end{equation}

Эта формула является окончательной. 
Для получения удельной теплоты испарения следует разделить полученное выражение на молярную массу $\mu$.

В нашем опыте температура жидкости измеряется термометром, давление пара определяется при помощи манометра, а производные $dP/dT$ или $d(\ln P)/d(1/T)$ находятся графически как угловой коэффициент касательной к кривой $P(T)$ или соответственно к кривой, у которой по оси абсцисс отложено $1/T$, а по оси ординат $\ln P$.

\subsection*{Экспериментальная установка}
\begin{wrapfigure}{r}{0.4\linewidth}
    \centering
    \includegraphics[scale=0.7]{scheme1.jpeg} 
    \caption{Схема установки}
\end{wrapfigure}
Схема
установки изображена на рис. 1. 
Наполненный водой резервуар 1 играет роль термостата. 
Нагревание термостата производится спиралью 2, подогреваемой электрическим током. 
Для охлаждения воды в термостате через змеевик 3 пропускается водопроводная вода. 
Вода в термостате перемешивается воздухом, поступающим через трубку 4. 
Температура воды измеряется термометром 5. 
В термостат погружён запаянный прибор 6 с исследуемой жидкостью. 
Над ней находится насыщенный пар (перед заполнением прибора воздух из него был откачан). 
Давление насыщенного пара определяется по ртутному манометру, соединённому с исследуемым объёмом. 
Отсчёт показаний манометра производится при помощи микроскопа.

\begin{figure}[htp]
    \centering
    \includegraphics[scale=0.7]{scheme2.jpeg}
    \caption{Схема установки}
\end{figure}

На рис. 2 приведена более полная схема такой же установки, но с использованием современного термостата. 
Установка включает термостат A, экспериментальный прибор B и отсчётный микроскоп C. 
Экспериментальный прибор B представляет собой ёмкость 12, заполненную водой. 
В неё погружён запаянный прибор 13 с исследуемой жидкостью 14. 
Перед заполнением исследуемой жидкости воздух из запаянного прибора был удалён, так что над жидкостью находится только её насыщенный пар. 
Давление пара определяется по ртутному манометру 15, соединённому с ёмкостью 13. 
Численная величина давления измеряется по разности показаний отсчётного микроскопа 16, настраиваемого последовательно на нижний и верхний уровни столбика ртути манометра. 
Показания микроскопа снимаются по шкале 17.

Описание прибора указывает на второе важное преимущество предложенного косвенного метода измерения $L$ перед прямым.
При непосредственном измерении теплоты испарения опыты нужно производить при неизменном давлении, и прибор не может быть запаян. 
При этом невозможно обеспечить такую чистоту и неизменность экспериментальных условий, как при нашей постановке опыта.

Описываемый прибор обладает важным недостатком: термометр определяет температуру термостата, а не исследуемой жидкости (или её пара). 
Эти температуры близки друг к другу лишь в том случае, если нагревание происходит достаточно медленно. 
Убедиться в том, что темп нагревания не является слишком быстрым, можно сравнивая результаты, полученные при нагревании и при остывании прибора. 
Такое сравнение необходимо сделать. 
Для ориентировки укажем, что температуру воды в калориметре следует менять не быстрее, чем на 1 $^oC$ в течение 1–3 минут.

\subsection*{Результаты измерений и обработка данных}
\begin{enumerate}
    \item Измерим разность уровней в ртутном U-образном манометре с помощью микроскопа и температуру по термометру или индикаторному табло.
    
    $T_0 = 22.31^oC,\ h_1 = 92.4 \pm 0.1\ мм, h_2 = 72.9\pm 0.1\ мм \Rightarrow \Delta h = 19.5\pm 0.1\ мм.$
    \item Включим термостат. Будем подогревать воду в калориметре, пропуская ток через нагреватель и будем следить за тем, чтобы воздух всё время перемешивал воду.
    При работе через каждый градус будем измерять давление и температуру (нагревать жидкость будем до $40 ^oC$). Данные занесём в табл. 1.
    
    \item Построим графики в координатах $T,\ P$ (рис. 3) и в координатах $1/T$, $\ln P$ (рис. 4).
    На графики нанесём точки, полученные при нагревании жидкости.

    По формуле (4) вычислим $L$, пользуясь данными, полученными сначала из одного, а потом из другого графика. 
    \begin{equation*}
        {dP \over dT} = k_1 = (256 \pm 8)\ Па/K \Rightarrow L_1 = (39.5 \pm 1.2)\ кДж/моль
    \end{equation*}

    \begin{equation*}
        {d (\ln P/P_0) \over dT} = k_2 = (-0.52 \pm 0.01)\ 10^4 K^{-1} \Rightarrow L_2 = (43.2 \pm 0.1)\ кДж/моль
    \end{equation*}

    \newpage
    \begin{table}[htp]
        \centering
        \begin{tabular}[]{|c|c|c|c|c|c|c|c|c|c|}
            \hline
            $T,\ ^oC$ & $h_1,\ мм$ & $h_2,\ мм$ & $\Delta h,$ & $\sigma_h,$ & $P,$ & $\sigma_p,$ & $\frac 1 T,$& $\ln{\frac P {P_0}}$ & $\sigma_{\ln p}$\\
            & & & $мм$ & $мм$ & $кПа$ & $кПа$ & $10^4K^{-1}$ & & \\
            \hline
            23 & 92.7 & 72.9 & 19.8 & 0.1 & 2.64 & 0.26 & 33.8 & 2.99 & 0.01 \\
            \hline
            24 & 93.9 & 71.5 & 22.4 & 0.1 & 2.99 & 0.3 & 33.7 & 3.11 & 0.01 \\
            \hline
            25 & 94.6 & 71.0 & 23.6 & 0.1 & 3.15 & 0.31 & 33.5 & 3.16 & 0.01 \\
            \hline
            26 & 95.2 & 70.6 & 24.6 & 0.1 & 3.28 & 0.33 & 33.4 & 3.2 & 0.01 \\
            \hline
            27 & 95.8 & 70.0 & 25.8 & 0.1 & 3.44 & 0.34 & 33.3 & 3.25 & 0.01 \\
            \hline
            28 & 96.5 & 69.1 & 27.4 & 0.1 & 3.65 & 0.37 & 33.2 & 3.31 & 0.01 \\
            \hline
            29 & 97.7 & 68.5 & 29.2 & 0.1 & 3.89 & 0.39 & 33.1 & 3.37 & 0.01 \\
            \hline
            30 & 98.3 & 67.5 & 30.8 & 0.1 & 4.11 & 0.41 & 33.0 & 3.43 & 0.01 \\
            \hline
            31 & 99.0 & 66.7 & 32.3 & 0.1 & 4.31 & 0.43 & 32.9 & 3.48 & 0.01 \\
            \hline
            32 & 100.0 & 65.8 & 34.2 & 0.1 & 4.56 & 0.46 & 32.8 & 3.53 & 0.01 \\
            \hline
            33 & 101.3 & 64.4 & 36.9 & 0.1 & 4.92 & 0.49 & 32.7 & 3.61 & 0.01 \\
            \hline
            34 & 102.2 & 64.1 & 38.1 & 0.1 & 5.08 & 0.51 & 32.6 & 3.64 & 0.01 \\
            \hline
            35 & 103.3 & 62.9 & 40.4 & 0.1 & 5.39 & 0.54 & 32.5 & 3.7 & 0.01 \\
            \hline
            36 & 104.5 & 61.8 & 42.7 & 0.1 & 5.69 & 0.57 & 32.3 & 3.75 & 0.01 \\
            \hline
            37 & 105.7 & 60.0 & 45.7 & 0.1 & 6.09 & 0.61 & 32.2 & 3.82 & 0.01 \\
            \hline
            38 & 107.0 & 58.9 & 48.1 & 0.1 & 6.41 & 0.64 & 32.1 & 3.87 & 0.01 \\
            \hline
            39 & 108.6 & 57.5 & 51.1 & 0.1 & 6.81 & 0.68 & 32.0 & 3.93 & 0.01 \\
            \hline
            40 & 109.9 & 56.5 & 53.4 & 0.1 & 7.12 & 0.71 & 31.9 & 3.98 & 0.01 \\
            \hline
        \end{tabular}
        \caption{Давление при повышении температуры}
    \end{table}

    \begin{figure*}[ht]
        \begin{flushright}
            \includegraphics[scale=0.9]{sign.png}
        \end{flushright}     
    \end{figure*}

    \begin{figure}[ht]
        \centering
        \includegraphics[scale=0.9]{PT2.png}
        \caption{Касательная при T $\approx$ 307К}
    \end{figure}

    \begin{figure}[htp]
        \centering
        \includegraphics[scale=0.9]{lnpt2.png}
        \caption{Аппроксимация зависимости $\ln{P}(1/T)$}
    \end{figure}

   
\end{enumerate}

\subsection*{Вывод}
В ходе данной работы подтверждена зависимость давления насыщенного пара от температуры.
Двумя способами при помощи уравнения Клайперона-Клаузиуса по полученным данным расчитаны значения теплоты испарения воды.
Первый способ основывался на исследовании зависимости P(T) и полученное значение равно $L_1 = (39.5 \pm 1.2)$ кДж/моль = ($2194 \pm 66$) кДж/кг ($\varepsilon = 3\%$).
Второй способ строился на зависимости $\ln P(1/T)$ и $L_2 = (43.2 \pm 0.1)$ кДж/моль = ($2400\pm 6$) кДж/кг ($\varepsilon = 1\%$).
Полученные значения в пределах погрешности совпадают с таблиным значением $L = 2259$ кДж/кг.
Последний способ более удобен для исследования, также его погрешность меньше в связи с тем, что в итоговую формулу входит меньше изменяемых параметров.
\end{document}