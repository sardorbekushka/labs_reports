\documentclass[12pt,a4paper]{article}
\usepackage{amsmath}
\usepackage{mathtext}
\usepackage{icomma}
\usepackage{amsfonts}
\usepackage{amssymb}
\usepackage[utf8]{inputenc}
\usepackage[T1,T2A]{fontenc}
\usepackage[english, russian]{babel}
\usepackage{graphicx}
\usepackage[left=2cm,right=2cm,top=2cm,bottom=2cm]{geometry}
\usepackage{calc}
\usepackage{wrapfig}
\usepackage{setspace}
\usepackage{indentfirst}
\usepackage{subfigure}
\usepackage[table,xcdraw]{xcolor}


\title{
Отчет о выполнении лабораторной работы 2.5.1 \\
Измерение коэффициента оверхностного натяжения жидкости.}

\author{Исламов Сардор, группа Б02-111}
\date{22 февраля 2022 г. }

\begin{document}
\maketitle

\subparagraph*{Аннотация.} В ходе работы измерена температурная зависимость коэффициента поверхностного натяжения дистиллированной воды с использованием известного коэффициента поверхностного натяжения спирта. Также определена полная поверхностная энергия и теплота, необходимая для изотермического образования единицы  поверхности жидкости при различной температуре. 

\subsection*{Теоретическое введение} 
Наличие поверхностного слоя приводит к различию давлений по разные стороны от искривленной границы раздела двух сред.  Для сферического пузырька с воздухом  внутри жидкости избыточное давление даётся формулой Лапласа:
\begin{equation}
    \Delta P = P_{внутри} - Р_{снаружи} = \frac {2 \sigma} r,
\end{equation}
где $\sigma$ – коэффициент поверхностного натяжения, $P_{внутри}$ и $Р_{снаружи}$ – давление внутри пузырька и снаружи, $r$ – радиус кривизны поверхности раздела двух фаз. Эта формула лежит в основе предлагаемого метода определения коэффициента поверхностного натяжения жидкости. Измеряется давление $\Delta P$, необходимое для выталкивания в жидкость пузырька воздуха.

\subsection*{Экспериментальные методы}
\subparagraph*{В работе используются:}прибор  Ребиндера  с термостатом и микроманометром; исследуемые жидкости; стаканы.
\\

Схема установки приведена на рис. 1.
\\

\begin{figure}[ht]
    \centering
    \includegraphics[width=0.7\linewidth]{image.jpeg}
    \caption{Схема установки}
    \label{fig:my_label}
\end{figure}

Исследуемая жидкость (дистиллированная вода) наливается в сосуд (колбу) В (рис.1). Тестовая жидкость (этиловый спирт) наливается  в сосуд Е. При измерениях колбы герметично закрываются  пробками. Через одну из двух пробок проходит полая металлическая игла С. Этой пробкой закрывается сосуд, в котором проводятся измерения. Верхний конец иглы открыт в атмосферу, а нижний погружен в жидкость. Другой сосуд герметично закрывается второй пробкой. При создании достаточного разрежения воздуха в колбе с иглой пузырьки воздуха начинают пробулькивать через жидкость. Поверхностное натяжение можно определить по величине разряжения $\Delta P$ (1), необходимого для прохождения пузырьков (при известном радиусе иглы).

Разрежение в системе создается с помощью аспиратора А. Кран $К_2$ разделяет две полости аспиратора. Верхняя полость при закрытом кране $К_2$ заполняется водой. Затем кран $К_2$ открывают и заполняют водой нижнюю полость аспиратора. Разрежение воздуха происходит в нижней полости при открывании крана $К_1$, когда вода вытекает из него по каплям. В колбах В и С, соединённых трубками с нижней полостью аспиратора, создается такое же пониженное давление. Разность давлений в полостях с разреженным воздухом и атмосферой измеряется спиртовым микроманометром.

Для стабилизации температуры исследуемой жидкости через рубашку D колбы В непрерывно прогоняется вода из термостата.


Обычно кончик иглы лишь касается поверхности жидкости, чтобы исключить влияние гидростатического давления столба жидкости. Однако при измерении температурной зависимости коэффициента поверхностного натяжения возникает ряд сложностей. Во-первых, большая теплопроводность металлической трубки приводит к тому, что температура на конце трубки заметно ниже, чем в глубине жидкости. Во-вторых, тепловое расширение поднимает уровень жидкости при увеличении температуры. 

Обе погрешности можно устранить, погрузив кончик трубки до самого дна. Полное давление, измеренное при этом микроманометром, $P = \Delta P + \rho gh$. Заметим, что $\rho gh$ от температуры практически не зависит, так как подъём уровня жидкости компенсируется уменьшением её плотности (произведение $\rho h$ определяется массой всей жидкости и поэтому постоянно). Величину  $\rho gh$ следует измерить двумя способами. Во-первых, замерить величину $Р_1 = \Delta P'$, когда кончик трубки только касается поверхности жидкости. Затем при этой же температуре опустить иглу до дна и замерить $Р_2= \rho gh + \Delta P''\ (\Delta P',\ \Delta ''$ – давление Лапласа). Из-за  несжимаемости  жидкости можно положить $\Delta P'= \Delta P''$ и тогда $\rho gh = Р_2-Р_1$. Во-вторых, при измерениях $Р_1$ и $Р_2$ замерить линейкой глубину погружения иглы $h$. Это можно сделать, замеряя расстояние между верхним концом иглы и любой неподвижной частью прибора при положении иглы на поверхности и в глубине колбы.

\subparagraph*{Замечание.}Чувствительность микроманометра высока, поэтому правильность его работы существенно зависит от правильности эксплуатации прибора. Все изменения в установке необходимо проводить, предварительно поставив переключатель микроманометра на атмосферу.

В частности, подобную же операцию необходимо сделать и при заполнении водой аспиратора А. В противном случае при заполнении аспиратора водой давление воздуха в системе повышается, спирт из трубки микроманометра выдавливается, в узлах соединений микроманометра образуются воздушные пузыри. Наличие этих пузырей приводит к полному нарушению калибровки манометра и невоспроизводимости измерений.

\subsection*{Методика измерений и обработка данных}
\begin{enumerate}
    \item Проверим герметичность установки. Для этого заполним аспиратор водой. Чистую сухую иглу установим в сосуд со спиртом так, чтобы кончик иглы лишь касался поверхности спирта. Плотно закроем обе колбы В и Е пробками. Откроем кран $К_1$ аспиратора и добьемся пробулькивания пузырьков воздуха в колбе. Замерим показания микроманометра. Закроем кран $К_1$. Наблюдаем за показаниями манометра: при отсутствии течи в установке столбик спирта в манометре неподвижен. 
    
    \item Убедившись в герметичности системы, начнем измерения. Откроем кран $К_1$. Подберем частоту падения капель из аспиратора так, чтобы максимальное давление манометра не зависело от этой частоты (не чаще, чем 1 капля в 5 секунд).

    \item Измерим максимальное давление $\Delta P_{спирт}$ при  пробулькивании пузырьков воздуха через спирт. По разбросу результатов оценим случайную погрешность измерения. Пользуясь табличным значением коэффициента поверхностного натяжения спирта, определим по формуле (1) диаметр иглы. Сравним полученный результат с диаметром иглы, измеренным по микроскопу.
    
    \begin{table}[ht]
        \centering
        \begin{tabular}{|c|c|c|c|}
            \hline
            $\Delta P$, Па & $\sigma_P$, Па & $d$, мм & $\sigma_d$, мм \\
            \hline
             80.41453 & 1.96133 & 1.13142& 0.03033\\
             \hline
        \end{tabular}
        \caption{Диаметр иглы при измерении $\Delta P$}
        \label{tab:my_label}
    \end{table}
    Разброс в показаниях отсутствует.
    Табличное значение $\sigma_{спирт} $ при $T=20^oC$ равно $22.8 \cdot 10^{-3}\ Н/м\Rightarrow d = (1.13142 \pm 0.03033)\ мм$.
    
    Значение диаметра иглы, полученное через микроскоп, в пределах погрешности совпадает с расчитанным: $d = (1.150 \pm 0.025)\ мм$.
    
    \item Перенесем предварительно промытую и просушенную от спирта иглу в колбу с дистиллированной водой. Измерим максимальное давление $Р_1$ при пробулькивании пузырьков, когда игла лишь касается поверхности воды. Аспиратор должен быть предварительно  заполнен водой почти доверху. Отрегулируем скорость поднятия уровня спирта в манометре и сохраним её в течение всех экспериментов. Измерим расстояние между верхним концом иглы и любой неподвижной частью прибора $h_1$.
    
    \item Утопим иглу до предела (между концом иглы и дном необходимо оставить небольшой зазор, чтобы образующийся пузырёк не касался дна). Измерим $h_2$ (как в пункте 4) и максимальное давление в пузырьках $Р_2$.
    По разности давлений $\Delta Р = Р_2 - Р_1$ определим глубину погружения $\Delta h$ иглы и сравним с $\Delta h =  h_1- h_2$.
    
    \begin{table}[ht]
        \centering
        \begin{tabular}{|c|c|c|c|c|c|c|}
            \hline
             $P_1$, Па & $P_2$, Па & $\sigma_P$, Па & $\Delta P$, Па & $\sigma_{\Delta P}$, Па & $\Delta h$, см & $\sigma_{\Delta h}$,  см \\
             \hline
             198.09433& 384.42068& 1.96133 & 186.32635&3.92266 & 1.90& 0.04\\
             \hline 
        \end{tabular}
        \caption{Глубина погружения через разницу давлений}
    \end{table}
    
    \begin{table}[ht]
        \centering
        \begin{tabular}{|c|c|c|c|c|}
        \hline
            $h_1$, см & $h_2$, см & $\sigma_h$, см & $\Delta h$, см & $\sigma_{\Delta h}$, см  \\
            \hline
             2.650& 0.950 & 0.025&1.70 & 0.05\\
            \hline
        \end{tabular}
        \caption{Измерение глубины погружения}
    \end{table}

    Полученные разными методами данные о глубине погружения иглы несколько отличаются друг от друга. В дальнейшем в работе будет использоваться значение $d = (1.70\pm 0.05)\ см$, полученное прямым измерением.
    
    \item Снимем температурную зависимость $\sigma (Т)$ дистиллированной воды. 
    Для этого включим термостат и подождем, пока нужная вам температура не стабилизируется. 
    Следует заметить, что термометр показывает температуру воды в термостате. 
    После этого проведем измерение давления. 
    Для уменьшения погрешности опыта замер давления при фиксированной температуре проводим несколько раз. 
    Проводить измерение температурной зависимости  рекомендуется в диапазоне  $20^oС - 60^oС$ через $5^oC$. 
    Запрещается нагревать воду в термостате выше $60^oC$.
    
    \item Оценим погрешность измерения давления и температуры. Рассчитаем величину коэффициента поверхностного натяжения воды $\sigma (T)$, используя значение диаметра иглы, полученное при измерениях на спирте (или измеренное на микроскопе).
    
    \begin{table}[ht]
        \centering
        \begin{tabular}{|c|c|c|c|c|c|}
        \hline
            $T,\ ^oC$ & $\sigma_T,\ ^oC$ & $\Delta P_2$, Па & $\sigma_P$, Па & $\sigma(T),\ 10^{-3}$, Н/м & $\sigma_{\sigma},\ 10^{-3}$ Н/м\\
            \hline
            23.5 & 0.5& 388.34334 & 1.96133&62.68924 & 4.21109\\
            \hline
            30.1 & 0.5& 388.34334 & 1.96133& 62.68924& 4.21109\\
            \hline
            35.0 &0.5 & 386.38021 & 1.96133& 62.13395& 4.19621\\
            \hline
            40.3 &0.5 & 384.42068 &1.96133 & 61.57969& 4.18136\\
            \hline
            45.2 & 0.5& 382.45935 &1.96133 & 61.02492& 4.16650\\
            \hline
            50.5 & 0.5& 380.49802 & 1.96133& 60.47015& 4.15163\\
            \hline
            55.0 &0.5 &378.53669 &1.96133 & 59.91538& 4.13677\\
            \hline
            59.9 &0.5 & 374.61403& 1.96133& 58.80583& 4.10704\\
            \hline
        \end{tabular}
        \caption{Измерение коэффициента поверхностного натяжения воды}
    \end{table}
    
    \item Построем график зависимости $\sigma (T)$ (рис. 2) и определим по графику температурный коэффициент $d\sigma / d T$. Оценим точность результата.
    
    По МНК получаем, что коэффициент наклона $\frac {d\sigma}{dT} = (-0.107\pm 0.009)\ \frac {мН}{м\cdot К}$
    \begin{figure}[ht]
    \centering
    \includegraphics[width=0.6\linewidth]{sigma_t.png}
    \caption{График зависимости $\sigma(T)$}
    \end{figure}
    \item На другом графике (рис. 3) построим зависимость от температуры
    
    а) теплоты образования единицы поверхности жидкости $q = -T\cdot d\sigma/dT$ и 
    
    б) поверхностной энергии $U$ единицы площади $F$:\ $U/F = (\sigma - T \cdot d \sigma/dT)$. 
    
    Значение $U/F$ постоянно и равно $(96.575 \pm 0.103)\ \frac {мН}{м\cdot К}$

    % Если построить все графики на одной плоскости, становится видно, что $U/F$ является 
    
    \begin{figure}[ht]
        \centering
        \includegraphics[width=0.6\linewidth]{q_UF_t.png}
        \caption{Графики зависимости $q(T)$ и $U/F (T)$}
    \end{figure}
\end{enumerate}

\subsection*{Вывод}
В данной работе двумя методами (прямыми измерениеми и расчетами) были получены диаметр иглы и глубина ее погружения при проведении эксперимента. Косвенный расчет диаметра основывается на известном коэффиценте поверхностного натяжения спирта.
Значения диаметров $d_1 =(1.13142\pm 0.03033)\ мм$ и $d_2 = (1.150 \pm 0.025)\ мм$ совпадают в пределах погрешности. Глубина погружения же несколко отличается: $\Delta h_1 = (1.90 \pm 0.04)\ см$, $\Delta h_2 = (1.70\pm 0.05)\ см$, что может быть связано с неточностью снятия показаний давления и неидеальностью установки. 
На основе полученных данных построен график зависимости поверхностного натяжения воды от температуры и методом наименьших квадратов определен коэффициент $\frac {d\sigma}{dT} = (-0.107\pm 0.009)\ \frac {мН}{м\cdot К}$. 
Также построены графики зависимости от температуры теплоты образования единицы поверхности жидкости $q = -T\cdot d\sigma/dT$ и поверхностной энергии $U$ единицы площади $F$:\ $U/F = (96.575 \pm 0.103)\ \frac {мН}{м\cdot К}$. 
Полученный коэффициент поверхностного натяжения при комнатной температуре $\sigma \approx 63 \ \frac {мН}{м\cdot К} $ не совпадает с табличным $\sigma \approx 73\ \frac {мН}{м\cdot К}$. 
Это может быть связано с причиной, описанной выше и некоторыми тонкостями работы с установкой: в ходе выполнения эксперимента оказалось, что крышка с иглой неисправна и при попытке ее опустить прокручивается на месте.
В связи с этим лаборант предпринял несколько попыток устранения неполадки.
В итоге игла была заменена, а часть воды из колбы выкачана, т.к. игла, даже находясь в крайнем верхнем положении, была погружена в воду на $1\div2\ мм$.
\end{document}