\documentclass[12pt,a4paper]{article}
\usepackage{amsmath}
\usepackage{mathtext}
\usepackage{icomma}
\usepackage{amsfonts}
\usepackage{amssymb}
\usepackage[utf8]{inputenc}
\usepackage[T1,T2A]{fontenc}
\usepackage[english, russian]{babel}
\usepackage{graphicx}
\usepackage[left=2cm,right=2cm,top=2cm,bottom=2cm]{geometry}
\usepackage{calc}
\usepackage{wrapfig}
\usepackage{setspace}
\usepackage{indentfirst}
\usepackage{subfigure}
\usepackage[table,xcdraw]{xcolor}


\title{
Отчет о выполнении лабораторной работы 2.1.6 \\
Эффект Джоуля-Томсона.}

\author{Исламов Сардор, группа Б02-111}
\date{14 марта 2022 г.}

\begin{document}

\maketitle

\subparagraph*{Аннотация.} Определено изменение температуры углекислого газа при протекании через малопроницаемую перегородку при разных начальных значениях давления и температуры;
 по результатам опытов вычислены коэффициенты Ван-дер-Ваальса «a» и «b».

\subsection*{Теоретическое введение}

Эффектом Джоуля–Томсона называется изменение температуры газа, медленно протекающего из области высокого в область низкого давления в условиях хорошей тепловой изоляции. 
В разреженных газах, которые приближаются по своим свойствам к идеальному газу, при таком течении температура газа не меняется. 
Эффект Джоуля–Томсона демонстрирует отличие исследуемого газа от идеального.

Рассмотрим стационарный поток газа между произвольными сечениями I и II трубки (до перегородки и после неё).
Пусть, для определённости, через трубку прошёл 1 моль углекислого газа; $\mu$ — его молярная масса. 
Молярные объёмы газа, его давления и отнесённые к молю внутренние энергии газа в сечениях I и II обозначим соответственно $V_1,\ P_1,\ U_1\ и\ V_2,\ P_2,\ U_2$. 
Для того чтобы ввести в трубку объём $V_1$, над газом нужно совершить работу $A_1 = P_1V_1$. 
Проходя через сечение II, газ сам совершает работу $A_2 = P_2V_2$. 
Так как через боковые стенки не происходит ни обмена теплом, ни передачи механической энергии, то
\begin{equation}
    A_1 - A_2 = \left( U_2 + \frac{\mu v_2 ^2}{2}\right) - \left(U_1 + \frac{\mu v_1^2}{2}\right)
\end{equation}

В уравнении (1) учтено изменение как внутренней (первые члены в скобках), так и кинетической (вторые члены в скобках) энергии газа. 
Подставляя в (1) написанные выражения для $A_1$ и $A_2$ и перегруппировывая члены, найдём
\begin{equation}
    H_1 - H_2 = (U_1 + P_1 V_1) - (U_2 + P_2 V_2) = \frac 1 2\mu (v_2^2 - v_1^2).
\end{equation}
Сделаем несколько замечаний. 
Прежде всего отметим, что в процессе Джоуля–Томсона газ испытывает в пористой перегородке существенное трение, приводящее к её нагреву. 
Потери энергии на нагрев трубки в начале процесса могут быть очень существенными и сильно искажают ход явления. 
После того как температура трубки установится и газ станет уносить с собой все выделенное им в пробке тепло, формула (1) становится точной, если, конечно, теплоизоляция трубки достаточно хороша и не происходит утечек тепла наружу через её стенки.

Второе замечание связано с правой частью (2). 
Процесс Джоуля–Томсона в чистом виде осуществляется лишь в том случае, если правой частью можно пренебречь, т. е. если макроскопическая скорость газа с обеих сторон трубки достаточно мала. 
У нас сейчас нет критерия, который позволил бы установить, когда это можно сделать. 
Поэтому мы отложим на некоторое время обсуждение вопроса о правой части (2), а пока будем считать, что энтальпия газа не меняется в соответствии с формулой.
В соответствии с этим получаем
\begin{equation}
    \mu_{д-т} = \frac {\Delta T}{\Delta P} \approx \frac {\frac {2a}{RT} - b}{C_p}
\end{equation}

Из формулы (3) видно, что эффект Джоуля–Томсона для не очень плотного газа зависит от соотношения величин a и b, которые оказывают противоположное влияние на знак эффекта. 
Если силы взаимодействия между молекулами велики, так что превалирует «поправка на давление», то основную роль играет член, содержащий $a$, и $$\frac {\Delta T} {\Delta P} > 0$$
т. е. газ при расширении охлаждается ($\Delta T < 0$, т. к. всегда $\Delta P < 0$).

В обратном случае (малые $a$) $$\frac {\Delta T} {\Delta P} < 0$$ т. е. газ нагревается. 
Как следует из формулы 3, при темепературе $T_i = \frac{2a}{Rb}$ коэффициент $\mu_{д-т}$ обращается в нуль.
Тогда по формулам связи параметров газа Ван-дер-Ваальса с критическими параметрами получаем:
\begin{equation}
    T_{инв} = \frac{27}4 T_{кр}.
\end{equation} 

При температуре Tинв эффект Джоуля–Томсона меняет знак: ниже температуры инверсии эффект положителен ($\mu_{д-т}$ > 0, газ охлаждается), выше $T_{инв}$ эффект отрицателен ($\mu_{д-т}$ < 0, газ нагревается).

Вернёмся к влиянию правой части уравнения (2) на изменение температуры расширяющегося газа. 
Для этого сравним изменение температуры, происходящее вследствие эффекта Джоуля–Томсона, с изменением температуры, возникающим из-за изменения кинетической энергии газа. 
Увеличение кинетической энергии газа вызывает заметное и приблизительно одинаковое понижение его температуры как у реальных, так и у идеальных газов. 
Поэтому при оценках нет смысла пользоваться сложными формулами для газа Ван-дер-Ваальса.

Заменяя в формуле (2) $U$ через $C_V T$ и $P V$ через $R T$, найдём
\begin{equation*}
    (R + C_V)(T_1 - T_2) = \mu \frac{v_2^2 - v_1^2}2
\end{equation*}
или 
\begin{equation*}
    \Delta T = \frac{\mu}{2C_p} (v_2^2-v_1^2).
\end{equation*}

В условиях нашего опыта расход газа $Q$ на выходе из пористой перегородки не превышает $10\ см^3/с$, а диаметр трубки равен $3\ мм$. Поэтому
\begin{equation*}
    v_2 \le \frac{4Q}{\pi d^2} = \frac{4\cdot 10\ см^3/c}{3.14 \cdot (0.9)^2\ см^2} \approx 140\ см/c.
\end{equation*}

Скорость $v_1$ газа у входа в пробку относится к скорости $v_2$ у выхода из неё как давление $P_2$ относится к давлению $P_1$. 
В нашей установке $P_1 = 4$ атм, a $P_2 = 1$ атм, поэтому

\begin{equation*}
    v_1 = \frac {P_2}{P_1}v_2 = \frac{1\ атм}{4\ атм} \cdot 140\ см/с=35\ см/с.
\end{equation*}

Для углекислого газа имеем $\mu = 44\ г/моль,\ C_p = 40 \frac{Дж}{моль\cdot K} \Rightarrow \Delta T = 7\cdot 10^{-4} K$
Это изменение температуры ничтожно мало по сравнению с измеряемым эффектом (несколько градусов).

В данной лабораторной работе исследуется коэффициент дифференциального эффекта Джоуля–Томсона для углекислого газа. 
По экспериментальным результатам оценивается коэффициент теплового расширения, постоянные в уравнении Ван-дер-Ваальса и температура инверсии углекислого газа. 
Начальная температура газа $T_1$ задаётся термостатом. 
Измерения проводятся при трёх температурах: комнатной, $30^oC$ и $40^oC$.

\subsection*{Экспериментальная установка}
\begin{figure}[htp]
    \centering
    \includegraphics[width=0.7\linewidth]{scheme.png}
    \caption{Схема установки}
\end{figure}
Схема установки для исследования эффекта Джоуля–Томсона в углекислом газе представлена на рисунке 1. 
Основным элементом установки является трубка 1 с пористой перегородкой 2, через которую пропускается исследуемый газ. 
Трубка имеет длину 80 мм и сделана из нержавеющей стали, обладающей, как известно, малой теплопроводностью. 
Диаметр трубки $d$ = 3 мм, толщина стенок 0,2 мм. Пористая перегородка расположена в конце трубки и представляет собой стеклянную пористую пробку со множеством узких и длинных каналов. 
Пористость и толщина пробки ($l$ = 5 мм) подобраны так, чтобы обеспечить оптимальный поток газа при перепаде давлений $\Delta P \le$ 4 атм (расход газа составляет около $10\ см^3/с$); 
при этом в результате эффекта Джоуля–Томсона создаётся достаточная разность температур.

Углекислый газ под повышенным давлением поступает в трубку через змеевик 5 из балластного баллона 6. 
Медный змеевик омывается водой и нагревает медленно протекающий через него газ до температуры воды в термостате. 
Температура воды измеряется термометром $T_в$, помещённым в термостате. 
Требуемая температура воды устанавливается и поддерживается во время эксперимента при помощи контактного термометра $T_к$.

Давление газа в трубке измеряется манометром М и регулируется вентилем В (при открывании вентиля В, т. е. при повороте ручки против часовой стрелки, давление $P_1$ повышается). 
Манометр М измеряет разность между давлением внутри трубки и наружным (атмосферным) давлением. Так как углекислый газ после пористой перегородки выходит в область с атмосферным давлением $P_2$, то этот манометр непосредственно измеряет перепад давления на входе и на выходе трубки $\Delta P = P_1 - P_2$.

Разность температур газа до перегородки и после неё измеряется дифференциальной термопарой медь — константан. 
Константановая проволока диаметром 0,1 мм соединяет спаи 8 и 9, а медные проволоки (того же диаметра) подсоединены к цифровому вольтметру 7. 
Отвод тепла через проволоку столь малого сечения пренебрежимо мал. 
Для уменьшения теплоотвода трубка с пористой перегородкой помещена в трубу Дьюара 3, стенки которой посеребрены, для уменьшения теплоотдачи, связанной с излучением. 
Для уменьшения теплоотдачи за счёт конвекции один конец трубы Дьюара уплотнен кольцом 4, а другой закрыт пробкой 10 из пенопласта. 
Такая пробка практически не создаёт перепада давлений между внутренней полостью трубы и атмосферой.

\subsection*{Результаты изменений и обработка данных}
\begin{enumerate}
    \item Запишем величину показаний вольтметра при $\Delta P = 0$. 
    Используем эту величину для корректировки показаний вольтметра в дальнейших измерениях: $\varepsilon = U(P) - U(0),\ U(0) = 0.005\ мВ$

    $$\alpha_{20^oC} = 40.7\ мкВ/{^oC},\ \alpha_{30^oC} = 41.6\ мкВ/{^oC},\ \alpha_{40^oC} = 42.5\ мкВ/{^oC}$$
    \item После установления избыточного давления $\Delta P \approx 4\ атм$ и прекращения переходных процессов (10-15 мин) запишем показания вольтметра.
    Далее будем снимать показания каждые 0.5 - 1 атм. Проведем измерения для нескольких значений температуры.
    
    $\sigma_p \approx 0.25/16\ Па \approx 0.015\ Па$, однако, в связи с тем, что разметка на манометре нестандартная, возможна дополниительная ошибка при выставлении значений.
    Возьмем $\sigma_p = 0.02\ Па$.

    \begin{equation*}
        \sigma_T = \Delta T \sqrt{\left({\sigma_p \over \Delta P}\right)^2 + \left({\sigma_u \over U}\right)^2}
    \end{equation*}
    
    Занесем полученные значения в таблицы.
    \begin{table}[htp]
        \centering
        \begin{tabular}[htp]{|c|c|c|c|c|c|}
            \hline
            \multicolumn{6}{|c|}{$T=20^oC$}\\
            \hline
            $\Delta P,\ атм$&$\sigma_p,\ атм$&$U,-1\ мВ$&$\sigma_U,\ мВ$&$\Delta T\ K$&$\sigma_{\Delta T},\ K$\\
            \hline
            4.00&0.02&0.150&0.001&3.56&0.18\\
            \hline
            3.00&0.02&0.103&0.001&2.41&0.16\\
            \hline
            2.50&0.02&0.084&0.001&1.94&0.15\\
            \hline
            2.00&0.02&0.063&0.001&1.43&0.14\\
            \hline
        \end{tabular}
    \end{table}

    \begin{table}[htp]
        \centering
        \begin{tabular}[htp]{|c|c|c|c|c|c|}
            \hline
            \multicolumn{6}{|c|}{$T=30^oC$}\\
            \hline
            $\Delta P,\ атм$&$\sigma_p,\ атм$&$U,-1\ мВ$&$\sigma_U,\ мВ$&$\Delta T\ K$&$\sigma_{\Delta T},\ K$\\
            \hline
            4.00&0.02&0.144&0.001&3.34&0.17\\
            \hline
            3.00&0.02&0.098&0.001&2.24&0.15\\
            \hline
            2.50&0.02&0.076&0.001&1.71&0.14\\
            \hline
            2.00&0.02&0.055&0.001&1.20&0.12\\
            \hline
        \end{tabular}
    \end{table}

    \begin{table}[htp]
        \centering
        \begin{tabular}[htp]{|c|c|c|c|c|c|}
            \hline
            \multicolumn{6}{|c|}{$T=40^oC$}\\
            \hline
            $\Delta P,\ атм$&$\sigma_p,\ атм$&$U,-1\ мВ$&$\sigma_U,\ мВ$&$\Delta T\ K$&$\sigma_{\Delta T},\ K$\\
            \hline
            4.00&0.02&0.134&0.001&3.04&0.15\\
            \hline
            3.00&0.02&0.091&0.001&2.02&0.14\\
            \hline
            2.50&0.02&0.072&0.001&1.58&0.13\\
            \hline
            2.00&0.02&0.051&0.001&1.08&0.11\\
            \hline
        \end{tabular}
    \end{table}

    \item Отложим точки на графике и по коэффициенту наклона найдём коэффициенты Джоуля-Томсона.
    \begin{figure}[htp]
        \centering
        \includegraphics[scale=0.7]{TP.png}
        \caption{График зависимости $\Delta T (\Delta K)$}
    \end{figure}

    \begin{equation*}
        \mu_1 = 0.83 \pm 0.03\ К/атм,\ \mu_2 = 0.76 \pm 0.04\ К/атм,\ \mu_3 = 0.69 \pm 0.04\ К/атм. 
    \end{equation*}
    \item По формуле (3) вычислим параметры $a$ и $b$ газа:
    \begin{equation*}
        \begin{cases}
            \displaystyle a = \frac{(\mu_1 - \mu_2)C_pRT_1T_2}{2(T_2-T_1)}\\
            \displaystyle b = \frac{C_p(T_2\mu_2 - T_1\mu_1)}{T_1-T_2}   
        \end{cases}
    \end{equation*}

    \begin{equation*}
        {\partial a \over \partial \mu_1} = {C_p R T_1 T_2 \over 2(T_2 - T_1)} = {a \over \mu_1 - \mu_2}
    \end{equation*}

    Аналогично для $\mu_2,\ \sigma_{\mu} << \sigma_{T} \Rightarrow \sigma_a = a\displaystyle \sqrt{{1 \over (\mu_1 - \mu_2)^2}\left(\sigma_{\mu_1}^2 + \sigma_{\mu_2}^2\right)}$
    $$\sigma_b = b {\sqrt{(T_2\sigma_{\mu_2})^2 + (T_1\sigma_{\mu_1})^2} \over T_1 \mu_1 - T_2 \mu_2}$$
    \begin{equation*}
        a_1 = 1.03\pm 0.71\ {Н \cdot м^4 \over моль^2},\ b_1 = 516.40 \pm 488.09 {см^3 \over моль}
    \end{equation*}

    Из этих расчётов видно, что значения совсем не совпадают с табличными (при $T = T_{кр} = 304.15$K, что примерно совпадает с температурами, для которых проводился расчёт) $a = 0.37\ {Н \cdot м^4 \over моль^2},\ b = 42.79 {см^3 \over моль}$, и погрешности велики, поэтому смысла проводить расчеты для второй пары значений нет.
\end{enumerate}

\subsection*{Вывод}
В ходе выполнения работы выяснилась неприменимость модели газа Ван-дер-Ваальса для описания процессов, происходящих в данном опыте с углексилым газом. 
Полученные значения коэффициентов $a = 1.03\pm 0.71\ {Н \cdot м^4 \over моль^2},\ b = 516.40 \pm 488.09 {см^3 \over моль}
$ в уравнении Ван-дер-Ваальса сильно разнятся с табличными значениями $a = 0.37\ {Н \cdot м^4 \over моль^2},\ b = 42.79 {см^3 \over моль}$. 
Также частично это может быть связано с тем, что баллон, использованный в работе, одновременно использовался для работы на другой установке.

\end{document}